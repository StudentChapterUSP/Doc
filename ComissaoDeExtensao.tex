\documentclass{article}
\usepackage[portuguese]{babel}
\usepackage[utf8]{inputenc}
\usepackage{yfonts}
\usepackage{amsfonts}
\usepackage[margin=2cm]{geometry}
\usepackage{amsmath}
\usepackage{float}

\setlength{\parskip}{10pt plus 1pt minus 1pt}



\begin{document}


\begin{titlepage}
  \begin{center}
    \vspace*{\fill}
    \begin{LARGE}
      Relatório anual do grupo de extensão SIAM Student Chapter\\[1.5cm]
    \end{LARGE}
    \vspace*{\fill}
    \vfill
    \begin{normalsize}
      ICMC - USP\\
      São Carlos, 06/05/2019
    \end{normalsize}
  \end{center}
\end{titlepage}




No ano de 2018, o grupo buscou se organizar melhor internamente,
estabelecendo um estatuto. Além disso, buscou-se continuar com o
projeto que tínhamos então, o {\it Círculos Matemáticos}, que consistia
em ir à escola estadual Ludgero Braga e tentar falar de matemática de
forma (supostamente) interessante e lúdica com os jovens lá presentes,
no sábado da família. Julgamos ter atingido cerca de 10 jovens com
esse projeto.

Tivemos muitas dificuldades em conseguir a atenção dos jovens e em
realizar as atividades que propunhamos. Notando isso, buscamos entender
melhor essas dificuldades no processo de aprendizado e buscamos algo
que poderia servir de base para um projeto de ensino, chegando no
final do ano, à teoria construtivista do Seymour Papert, que decidimos
estudar melhor para aplicar no (então) ano seguinte.

Paralelamente, no decorrer de 2018, o grupo realizou por algumas semanas uma série de
reuniões sobre matemática aplicada a biologia, em uma espécie de grupo
de estudos que culminou com o convite do professor Rafael Guido do
IFSC para uma palestra.

Mais para o final do ano, começamos também com o projeto de CineClube, no
qual exibimos um filme escolhido pelo grupo a cada duas
semanas (idealmente) e seguir com uma discussão do mesmo. Esse projeto tem o
objetivo de servir como um evento de integração do grupo e da
comunidade externa, proporcionando um momento de convivência informal.

Atualmente, o grupo conta com 4 membros, notadamente:

\begin{itemize}
\item Ramon Guedes, aluno de mestrado de engenharia biológica;
\item Éricles Lima, aluno de graduação em Matemática Aplicada;
\item Helen Picoli, aluna de graduação em Matemática Aplicada;
\item Fernanda Ueno, aluna especial, visando mestrado em Matemática Aplicada;
\end{itemize}

\end{document}
